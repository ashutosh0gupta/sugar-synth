% Success of formal methods
Formal methods have successful in applications to a vast range of
problems from the analysis of systems.
%
In particular,
the verification of hardware designs
using SAT solver based model-checking has been applied
to industrial-scale problems~\cite{biere1999symbolic2}.
%
The verification of software systems is a much harder problem, and
many methods like bounded model-checking~\cite{biere2003bounded},
abstract interpretation~\cite{lattice77}, and
counterexample guided abstraction refinement (CEGAR)~\cite{ClarkeCEGAR} have
been successful for the problem.
%

Synthesis of programs that implements a given behavior
has been a focus of research for a while~\cite{PnueliSynthesis}.
%
In recent years, the synthesis of programs using SAT/SMT solvers
has gained momentum.
%
The approach encodes the search of a program that exhibits a certain
behavior into a satisfiability problem.
%
The solvers attempt to find a solution of the satisfiability problem.
%
The solution is the synthesized program.
%
In~\cite{SrivastavaSynthesis,Solar-Lezama2005},
the method has been successfully applied
to find programs that satisfy quantified specifications like sorting
programs, and have missing ``holes'' and need proper implementations
for them.
%
In~\cite{exampleSynth},
a set of pairs of input and output examples is the specification of a program,
and the synthesis method searches for programs in a space defined by a
template~\cite{sygus}.


% Boolean networks

% Program synthesis
Since the formal methods have been exhibiting effectiveness in the vast
range of problems, the
systems biology community has been applying the methods for various
biological problems~\cite{fisher2007executable}.
The formal methods approach is distinctly different from the statistical
approach traditionally used in biology.
In formal methods, we aim to match precisely the expected behavior rather
than learn approximate artifacts from biological data.
The use of formal methods belongs to two broad categories:
the analysis of biological models and the synthesis of the models.

The key focus has been to model gene regulatory networks (GRNs), since
they are the core of the central dogma of biology.
Boolean networks are often used to model
GRNs to find stable points or attractors, i.e., nodes in the networks
where a system eventually reaches no matter where it starts~\cite{wang2012BooleanOverview}.
% In~\cite{boolean-networks-stabilty}, the networks are used to evaluate
% stability of GRNs under mutations.
The Boolean networks are not often sufficient to adequately model the behavior
of GRNs.
The continuous-time Markov chains (CTMCs) are used to bring in the aspects of
timing and probabilistic constraints to GRNs.
The transient behaviors of GRNs are also crucial for various aspects
of biology.
% For example, the decision making in some cells is based on
% response timing to stimulus~\cite{response-timing-paper}.
In~\cite{delayedCTMC}, a methods based on
model-checking is used to estimate the time evolution of the probability
distributions over states of GRNs.


There are many biological processes where we can observe the system behavior,
but the exact mechanisms of the processes are not known.
%
Recently, the methods for formal synthesis are finding their applications in biology
\cite{dunn2014defining, xuPluripotency, booleanModelKarp13, paoletti2014analyzing}.
%
In~\cite{koksal2013synthesis},
GRNs with unknown interactions are modeled using
a automata-like model with missing components
and the specification is the variations of the behavior of the system under mutations.
Their approach uses a counterexample-guided inductive synthesis (CEGIS) based algorithm~\cite{cegis}.
CEGIS is a framework for synthesis. It first finds a program that may satisfy `samples' from
the specification. If the synthesized program satisfies the specifications, the CEGIS terminates.
Otherwise, the method learns a new sample where the program violates the specifications.
It adds the sample to the set and goes to the next iteration.
Typically, the constraint solvers find the programs during intermediate steps of CEGIS by
solving the generated queries.
%
Our method follows a similar pattern, where
a set of sampled observations is a specification, i.e., output molecules,
and we need to find the governing programs, i.e., production rules.
%
In a subsequent work~\cite{fisher2015synthesising}, a Boolean network model is used,
the functions attached to the nodes are considered unknown, and
a different kind of data sets provides the behavior specification.
In this approach, their earlier method is adopted to work in the new situation.
%

As far as we know, there has been no similar computation based analysis of glycan production rules.
%
However, there has been a theoretical analysis of the properties of the production
rules in~\cite{Jaiman2018}. The work identifies the conditions for the production
of a finite set of molecules and the cases of unique production methods.
%
In our work, we are taking the computational approach of synthesis from formal methods
instead of looking for the theoretical conditions.

% 
% To resolve this paradox we borrow from the field of
% al-gorithmic self-assembly, which explores how small buildingblocks
% with stochastic local interactions assemble into various global
% configurations [23]–[25].  A central inverse problemin self-assembly
% is to design building blocks that assembleinto a target shape.  This
% process has been studied using thetheoretical framework of Wang tiles:
% square building blockswith colored sides that stick to one another
% along sides withmatching colors [26].  For efficient assembly of Wang
% tiles,their coloring must be cleverly chosen so the target shape isthe
% unique terminal output of the stochastic assembly process[27].   That
% is, their assembly must be algorithmic.  Glycans may be considered a
% natural realization of the Wang construct,with monomers acting like
% tiles whose stickiness is encodedby GTase enzymes.  If glycan
% biosynthesis could be madealgorithmic, cells could suppress unwanted
% byproducts andgenerate only desired glycan oligomers with high yield.



%--------------------- DO NOT ERASE BELOW THIS LINE --------------------------

%%% Local Variables:
%%% mode: latex
%%% TeX-master: "main"
%%% End:
