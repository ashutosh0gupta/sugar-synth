
In this section, we present the formal model for the synthesis problem.
We model glycan molecules and production rules as labeled trees. The glycan molecules are assembled by applying the production rules repeatedly.
Our synthesis problem reduces into finding the pieces of trees that represent the production rules.

Let $S$ be the set of sugar monomers that builds glycans,
the oligomer molecules. Each $s \in S$ is associated with arity $m$ 
(written $arity(s) = m$).
% , {\em i.e.}, the maximum number of children for $s$.
The children of the monomers are indexed. We refer to the $k$th child of $s$ for some $k \leq arity(s)$.
They correspond to bonds at specific positions in the monomers where children are connected.
Now we define the glycan molecules as labeled trees. Now onward we refer to the glycans simply as molecules.

\begin{df}
A {\em molecule} $m = (V,M,C,v_0)$ is a labeled tree, where 
$V$ is a set of nodes in the tree,
$M : V \maps S$ maps nodes to their label, 
$C : V \times \naturals \pmaps V$ maps the indexed children of nodes, and
$v_0 \in V$ is the root of $m$.
A molecule must respect the arity of monomers, i.e., if $M(v) = s$ and $C(v,n) = v'$ then $n \leq arity(s)$.
\end{df}
% \todo{Rewrite the following paragraph}
% \todo{definitions mixed with algorithms seems strange I think you should define all the formal stuff first and then present the algorithms}
Let us define notations related to the tree structure.
Let $\mathit{m = (V,M,C,v_0)}$ and $m' = (V',M',C',v_0')$ be molecules.
With an abuse of notation, we write $v \in m$ to denote $v \in V$.
For each $v \in m$, if $(v,n)$ is not in the domain of $C$, we write $C(v,n) = \bot$.
We assume that $C(v,0) = \bot$.
%and if $n > n' > 0$ and $C(v,n) \neq \bot$, then $C(v,n') \neq \bot$.
Let $NumberOfChildren(v)$ be equal to the number of $n$s such that $C(v,n) \neq \bot$. 
A node $v \in V$ is a {\em leaf} of $m$ if $C(v,n) = \bot$ for each $n$.
Let $depth(v)$ be the length of the path from $v_0$ to $v$.
A {\em branch} of $m$ is a path from $v_0$ to some leaf of $m$.
Let $height(m)$ be the length of the longest branch in $m$.
We define ancestor relation recursively as follows. Let $ancestor(m,v,0) = v$.
For for $d > 0$, let $ancestor(m,v,d) = ancestor(m,v', d-1)$ if $C(v',i) = v$ for some $i$.

Since we will be matching the parts of the trees and applying rules to expand them,
let us introduce notations for matching.
Let recursively-defined predicate $Match(m,v,m',v')$ state that 
$v \in m$, $v' \in m'$, $s = M(v) = M(v')$, and $Match( m, C(v,n), m', C(v',n) )$ for each $n \leq arity(s)$ such that $C(v,n) \neq \bot$.
In other words, the subtree in $m$ rooted at $v$ is embedded in $m'$ at node $v'$.
Let $subtree(m)$ be the set of molecules such that
$m' = (\_,\_,\_,v_0') \in subtree(m) \lequiv Match(m,v_0,m',v_0')$.
Let us also define a utility to copy a subtree of a molecule into another molecule.
Let $Copy(m,v)$ return a molecule $m'' = (V'',M'',C'',v_0'')$ such that 
$V''$ is a set of fresh nodes, $Match(m,v,m'',v_0'')$, and $Match(m'',v_0'',m,v)$.
Both the $Match$ conditions say that the trees rooted at
$v$ and $v_0''$ are identical.

Now we define the model of the production process of the molecules.
A production rule expands a molecule $m$ by attaching a new piece of tree at a node that has a vacant spot among its children if the surroundings of the
node satisfy some condition.
The rule is modeled as a tree that has two parts.
One part should already be there in $m$ and the other part will be appended to $m$.

\begin{df}
  A {\em production rule} $r = (V, M, C, v_0, v_e)$ is a labeled tree, where
  $V$ is a set of nodes,
  $M : V \maps S$ maps nodes to labels, 
  $C : V \times \naturals \pmaps V$ maps the indexed children of nodes,
  $v_0 \in V$ is the root, and
  $v_e \in V$ is the root of expanding part of the rule.
\end{df}

If we {\em apply}, a rule $r$ on a molecule $m$, then it is extended at some node
$v \in m$. A copy of the descendants of $v_e$ will be attached to
$v$ in $m$, and the rest of the nodes in the rule have to match $v$ and above.
We call the descendants of $v_e$ as {\em expanding nodes}
and all the other nodes as {\em matching nodes}.


\begin{wrapfigure}{r}{0.37\textwidth}
  \vspace{-12mm}
    % \begin{minipage}{0.48\linewidth}
      \small
    % \center
  \begin{tikzpicture}[shorten >=1pt,thick,node distance=1cm,on grid]
    \node[loc] (v0) {$A$};
    \node[loc, below right of=v0] (v1) {$B$};
    \node[sqloc, below left of=v0] (v2) {$A$};
    \path[->] (v0) edge (v1);
    \path[->] (v0) edge (v2);
  \end{tikzpicture}
  % \par
  % (a)
  % \end{minipage}
  % \begin{minipage}{0.48\linewidth}
    % \small
    % \center
  \quad
  \begin{tikzpicture}[shorten >=1pt,thick,node distance=1cm,on grid]
    \node[loc] (v0) {$A$};
    \node[loc, below right of=v0] (v1) {$A$};
    \node[loc, below right of=v1] (v3) {$B$};
    \node[loc, below left of=v1] (v2) {$A$};
    \path[->] (v0) edge (v1);
    \path[->,dashed] (v1) edge (v2);
    \path[->] (v1) edge (v3);
  \end{tikzpicture}
  % \par
  \hfill (a) \hfill (b)\hfill\mbox{}
  % (b)
  % \end{minipage}
  \caption{(a) A rule. (b) An application of the rule.}
  \label{fig:exrule}
  \vspace{-6mm}
\end{wrapfigure}
\paragraph{Example:} In Figure~\ref{fig:exrule}(a), we present a rule. It has
  two kinds of nodes.
  The rule adds the square node ($v_e$).
  The circular nodes %$A$ and its right child $B$
  are the pattern, which must be present in the molecule to apply the rule.
  In Figure~\ref{fig:exrule}(b), we present an application of the rule.
  The solid tree with three nodes is the initial molecule.
  The middle node $A$ and its right child $B$ forms a pattern, where the rule is applicable.
  %Upon applying the rule,
  The rule adds a left child with label $A$ to the middle node.
  The rule is not applicable at the root $A$ due to pattern mismatch.
  % , since it has a right child $A$
  % and the combination does not match the pattern.


We naturally extend the definitions related to molecules, including $Match$ and $Copy$,
to the production rules.
Let us formally define the molecule productions using the rules.
Let $m = (V,M,C,v_0)$ be a molecule and $r = (V_r, M_r, C_r, v_{0r}, v_e)$ be a production rule.
Let $d$ be such that $v_{0r} = ancestor(r,v_e,d)$, i.e., $v_e$ is at the depth $d$ in $r$.
Let $i$ be such that $C_r(v',i) = v_{0r}$ for some $v' \in r$.
We apply $r$ on $m$ at node $v \in m$ such that $C(v,i) = \bot$.
We obtain an expanded molecule as follows.
Let $(V',M',C',v_0') = Copy(r,v_e)$.
The expanded molecule is
$
m' = (V \uplus V', M \uplus M', C \uplus C' \uplus \{(v,i) \mapsto v_0'\}, v_0)
$ if
$Match( r, v_{0r}, m', ancestor(m', v_0', d) )$ where $\uplus$ is the disjoint union.
The match condition states that after attaching the new nodes $V'$
the rule tree must be embedded in $m'$ at the $d$th ancestor of $v'_0$. 
We write $m' = Apply(m, v, r)$ to indicate the application of $r$
on molecule $m$ at node $v$ that results in $m'$.
If $r$ is not applicable at $v$, we write $Apply(m, v, r) = \bot$.
We write $m' = Apply(m, r)$ if there is a $v \in m$ such that
$m' = Apply(m,v,r)$.

Let $R$ be a set of rules.
A molecule $m$ is {\em producible} by $R$ from a set of molecules $Q$
if there is sequence of molecules $m_0,...,m_k$
such that $m_0 \in Q$, $m_k=m$, and
for each $0<i\leq k$, $m_{i} = Apply(m_{i-1},r)$ for some $r \in R$.
Let $P(Q,R)$ denote the set of molecules that are producible from
rules $R$ from a set of molecules $Q$.
We have discussed in section~\ref{sec:bio} that all the production rules
are not applied at the same time.
The rules may live in compartments and
the rule sets of the compartments are applied one after another.
To model compartments for the rules,
let us suppose we have a sequence $R_1,...,R_k$ of set of rules.
Let $P(Q, R_1,..,R_k) = P(..P(P(Q,R_1),R_2),..,R_k)$ denoting the
trees obtained after applying the rule sets one after another.

% \subsection{Synthesis problem}
In nature, we observe a set of glycan molecules $\mu$ present in a cell.
%
However, we do not necessarily know the production rules to produce the molecules.
%
We will be developing a method to find the production rules.
%
The {\em synthesis problem} is to find a set $R$ of production rules
such that $\mu = P(S,R)$,
where $S$ is the set of monomers.

% %
% We can define a more general form of the synthesis problem.
% Let us suppose we are also given $k$ compartments with unknown rules.
% The goal of synthesis is to find  $R_1,...,R_k$ such that
% $\mu = P(S, R_1,..,R_k)$.
% We may relax the requirement of exactly producing $\mu$.
% We may say that molecules in $\mu$ have to be produced, but we are ok
% if subtrees of the molecules of $\mu$ are also produced.
% Formally, we may weaken the requirement to $\mu \subseteq P(Q, R_1,..,R_k) \subseteq subtree(\mu)$.

% The above are a few simplified versions of the biological problem.
% We may generalize the problem further where
% %rules are partitioned and the partitions are applied in phases one after another,
% rules are not applied exhaustively due to time constraints,
% the given $\mu$ is finite and may not be exhaustive,
% and we only know the weights of the parts of molecules in $\mu$.
% We will first present a method for solving a simplified version of the problem.
% However, our tool handles some of the above variations.
% We will discuss the variations in section~\ref{sec:variations}.

%--------------------- DO NOT ERASE BELOW THIS LINE --------------------------

%%% Local Variables: 
%%% mode: latex
%%% TeX-master: "main"
%%% End: 
