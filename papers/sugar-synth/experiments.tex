In this section, we  present our implementation of the proposed method for the synthesis
and discuss the interesting optimizations in the implementation.
%
We also apply our tool on several data sets and illustrate the usability of our method.

\subsection{Implementation} 
We have implemented method \textsc{SugarSynth} in a prototype tool {\ourtool} for
analyzing the performance of the technique in practice.
%
The tool, written in {\tt C++}, uses {\zthree}\cite{z3} as the SMT solver
to discharge the generated satisfiability queries.
%
We have developed a specialized file format for giving input to~\ourtool.
%
The format allows one to declare monomers, list molecules over the monomers
in the SMTLIB like format,
and pass the input parameters of the method.


\subsection{Optimizations}
We have incorporated several optimizations to gain efficiency or reduce
the number of iterations in the method. Let us discuss some of the remarkable
optimizations.

\paragraph{Ordered templates}
Our method constructs a list of templates.
Since the templates are symmetric, there is a potential of unnecessary
iterations.
Let us suppose the method rejects a set of learned rules.
In the subsequent queries include the constraints that reject the set of rules,
the solvers may choose the same set of rules again by permuting the mapping
from the templates to the learned rule.
To avoid these iterations, we define an arbitrary total order over the rules.
We add ordering constraints stating that any assignments to the list of templates should produce
ordered rules with respect to the total order.

\paragraph{Quantifier instantiations}
The constraints for rejecting the counterexample molecule is universally quantified at line~\ref{line:negCons} in algorithm~\ref{alg:sugar-synth}.
For any assignment, the solver needs to try sufficiently many instantiations of the quantifiers
to ensure that the query is not unsatisfiable.
The instantiations sufficiently slow down the solving process.
We assist the solver by also adding an instantiation of the quantifiers that is equal
to the values of the corresponding variables in the assignment $a$ at line~\ref{line:negModel}.
We observe that for some inputs only adding the instantiations and not the universally quantified
formula is more effective.

\paragraph{Constraints for counterexample molecules}
In the presentation of \textsc{SugarSynth}, we construct constraints of counterexample molecules
at line~\ref{line:consNewR} in each iteration.
The repeated work is impractical because the formula management system of Z3 will be overwhelmed by
the construction of many terms repeatedly. In our implementation, we construct the constraints
once outside the while loop.
We pass the templates as the second parameter instead of concrete rules to~\textsc{EncodeProduce}.
Later at the solving time in line~\ref{line:negModel}, constraints are added to assign values
of the template variables that were returned by the solver at line~\ref{line:posModel}.
Due to the assignments, the templates become concrete rules in the context of solving.


\subsection{Benchmark}

We have applied our tool on four sets of molecules form
respiratory mucins of a cystic fibrosis patient (D1),
human chorionic gonadotropin from a cancer cell line (D2), horse chorionic gonadotropin (D3), and
hydra (D4)~\cite{Jaiman2018,hydra}.
The availability of clean data, where we are clear about the source and the collection
methodology used, limits our choices for the data.

\subsection{Results}
\begin{table}[t]
  \centering
  \tiny
  \begin{minipage}{0.48\linewidth}
  \begin{tabular}[t]{|c|c|c|c|c|c|c|}\hline
     & \#mol- & \#Rules & Rule  & \#Comp- & success? & Time \\
         & ecules   &         & depth & artments &          & (in secs.) \\\hline
    %%cf-mucin.sugar%%
         &   & 7  & 3 & 1 & Yes &  3.02 \\\cline{3-7}
    D1   & 6 & 7  & 4 & 2 & Yes & {\bf 1.60}  \\\cline{3-7}
         &   & 6  & 3 & 3 & Yes & 9.36  \\\hline
         
    %%horse-cg.sugar%%
         &   & 7  & 3 & 2 & Yes & 14.37  \\\cline{3-7}
    D2   & 3 & 5  & 3 & 2 & Yes & {\bf 7.97}  \\\cline{3-7}
         &   & 5  & 3 & 3 & Yes  &  13.42 \\\hline
         
    %%sars-cov2.sugar%%     
         &   & 6  & 4  & 2  & Yes & 1.02  \\\cline{3-7}
    D3   & 6 & 5  & 2 & 1 & Yes & {\bf 0.57}  \\\cline{3-7}
         &   & 5  & 4 & 1 & Yes  &  0.71 \\\hline
         
    %%human-cg.sugar%%     
         &   & 8  & 4  & 1  & Yes & 4.35  \\\cline{3-7}
    D4   & 3 & 6  & 3 & 1 & Yes & {\bf 0.85}  \\\cline{3-7}
         &   & 6  & 2 & 2 & No  &  1.17 \\\hline
         
    % %%D4.sugar%     
    %      &   & 6  & 2 & 1 & No &  0.64 \\\cline{3-7}
    % D5   & 3 & 7  & 2 & 1 & Yes & {\bf 0.72}  \\\cline{3-7}
    %      &   & 8  & 4 & 1 & Yes  &  2.39 \\\hline
         
    % %%D5.sugar%     
    %      &   & 5  & 3 & 2 & Yes &  0.86 \\\cline{3-7}
    % D6   & 2 & 5  & 3 & 1 & Yes & {\bf 0.73}  \\\cline{3-7}
    %      &   & 4  & 2 & 1 & No  & 0.69  \\\hline
  \end{tabular}    
  \end{minipage}
  \begin{minipage}{0.48\linewidth}

  \begin{tabular}[t]{|c|c|c|c|c|c|c|}\hline
     & \#mol- & \#Rules & Rule  & \#Comp- & success? & Time \\
         & ecules   &         & depth & artments &          & (in secs.) \\\hline
    % %%cf-mucin.sugar%%
    %      &   & 7  & 3 & 1 & Yes &  3.02 \\\cline{3-7}
    % D1   & 6 & 7  & 4 & 2 & Yes & {\bf 1.60}  \\\cline{3-7}
    %      &   & 6  & 3 & 3 & Yes & 9.36  \\\hline
         
    % %%horse-cg.sugar%%
    %      &   & 7  & 3 & 2 & Yes & 14.37  \\\cline{3-7}
    % D2   & 3 & 5  & 3 & 2 & Yes & {\bf 7.97}  \\\cline{3-7}
    %      &   & 5  & 3 & 3 & Yes  &  13.42 \\\hline
         
    % %%sars-cov2.sugar%%     
    %      &   & 6  & 4  & 2  & Yes & 1.02  \\\cline{3-7}
    % D3   & 6 & 5  & 2 & 1 & Yes & {\bf 0.57}  \\\cline{3-7}
    %      &   & 5  & 4 & 1 & Yes  &  0.71 \\\hline
         
    % %%human-cg.sugar%%     
    %      &   & 8  & 4  & 1  & Yes & 4.35  \\\cline{3-7}
    % D4   & 3 & 6  & 3 & 1 & Yes & {\bf 0.85}  \\\cline{3-7}
    %      &   & 6  & 2 & 2 & No  &  1.17 \\\hline
         
    %%D4.sugar%     
         &   & 6  & 2 & 1 & No &  0.64 \\\cline{3-7}
    D5   & 3 & 7  & 2 & 1 & Yes & {\bf 0.72}  \\\cline{3-7}
         &   & 8  & 4 & 1 & Yes  &  2.39 \\\hline
         
    %%D5.sugar%     
         &   & 5  & 3 & 2 & Yes &  0.86 \\\cline{3-7}
    D6   & 2 & 5  & 3 & 1 & Yes & {\bf 0.73}  \\\cline{3-7}
         &   & 4  & 2 & 1 & No  & 0.69  \\\hline
         
    %%D7.sugar%     
         &   & 5  & 3 & 2 & Yes &  0.72 \\\cline{3-7}
    D7   & 3 & 5  & 3 & 1 & Yes & {\bf 0.65}  \\\cline{3-7}
         &   & 6  & 2 & 1 & No  & 0.69  \\\hline
         
    %%hydra.sugar%     
         &   & 4  & 3 & 2 & No &  0.79 \\\cline{3-7}
    D8   & 3 & 5  & 3 & 2 & Yes & {\bf 0.84}  \\\cline{3-7}
         &   & 8  & 4 & 3 & Yes  & 1.53   \\\hline
  \end{tabular}
    
  \end{minipage}
  \caption{Results of applying \ourtool on data sets.
    %Bold faced numbers are the most optimal timings.
  }
  \label{tab:results}
  % \vspace{-9mm}
\end{table}
%%% Local Variables:
%%% mode: latex
%%% TeX-master: "main"
%%% End:


We have applied \ourtool~on the data set. For each data set, we choose several
parameter combinations to illustrate the relative performance of the tool.
If we did not give large enough parameters, then the tool fails to synthesize the rules.

For D1, we synthesize the rules in 1.2 seconds. If we reduced any of the first
two parameters, we failed.
Giving an extra compartment in the first row did not impact the performance.
We synthesize the rules for D2, which is also our motivating example, in 0.4 seconds.
The performance of the tool is also robust against variations of the parameters.
We synthesize the rules for D3 in 0.9 seconds if we use optimal values for the parameter.
However, if we use too many resources or too little, the tool runs for a long time,
which is typical. The search in combinatorial space is highly sensitive to the parameters.

The synthesis for D4 takes a long time. We also observe the trade-off between the number of compartments
and the depth of the rules at the first and second row of D4.
For D3, we also learned large rules, i.e., they are adding many nodes at a single time.
Currently, the tool does not have any objective function to optimize the synthesized rules.
We are hoping to see more data sets such that we can develop a clearer picture
for a reasonable objective function.

% Our synthesized production rules for the first three data sets
% match with the reported rules in the literature.
% For the fourth, there is no hypothesis available to us.

The results suggest that the tool is potentially applicable to larger data sets with
the variants discussed earlier. However, we have not fully modeled all the information
and biological intuition --- for example, the objective function for the rules.

% As we have discussed earlier, we have implemented variants that encode 

%--------------------- DO NOT ERASE BELOW THIS LINE --------------------------

%%% Local Variables:
%%% mode: latex
%%% TeX-master: "main"
%%% End:
