% contributions
% 
% In this paper,
We have presented a novel method for synthesizing production rules for glycans.
\textsc{SugarSynth} is a CEGIS like iterative method.
We have applied our method on data sets to illustrate that our method
can find the rules in real-world scenarios.
Our method is the {\em first} to apply formal methods for the synthesis problem.
We have opened a new direction for the application of formal methods in biology.

% % Future extensions

% However, there are limitations to our current method.
% Apart from the case of repeating treelets,
% our method currently only operates on finite sets and objects.
% For example, we do not look for counterexample molecules for arbitrary heights.
% The design of the method has a flavor of bounded model-checking.
% A couple of parameters of the method impose the limit. 
% We are working to develop a method that is independent of the limitation.

% % Input data
% We do not support a full end to end interpretation of the data from the wet experiments.
% We assumed that we know the precise set of molecules as input.
% However, the wet biological experiments that identify glycans in a cell do not directly
% produce the set of molecules.
% The information about the glycans is limited.
% The experiments provide only the molecular weights of the glycans.
% Furthermore, they may selectively break bonds of glycans and read the weights of the
% parts.
% Using the information and computational methods, biologists estimate the likely
% set of observed glycans.
% The methods are not deterministic and may not report all possible combinations
% of molecules that may satisfy the given data.
% % The usual experimental errors may also cause failure to detect molecules.
% Much of human intuition guides the data interpretation.
% In our work, we have ignored the aspect of
% the interpretation step.
% In the future, we will try to bring this part of data analysis
% into our toolchain. 

% We are also considering to support a few more variants of the synthesis problem.
% For example, the boundary between the compartments may not be strict.
% On the transition from a compartment to next,
% the production rules of the first may not be immediately disabled and
% are phased out slowly, while the rules of the next compartment are already applicable.
% This momentary interleaving of the rules may explain the production of certain molecules.

% We may also consider the other stay models where a molecule may stay
% a limited amount of time and the number of operations on the molecules is
% bounded.
% In the model of stay, we may need to work out if we should bound the number
% of operations per branch or for the whole molecule.
% These kinds of modeling will require incorporation of time and probability in the method.

% We are also working to validate our synthesized production rules using wet experiments.
% We will be aiming for a cell that produces a large number of glycans and
% does not immediately lend itself to a human guess for the production rules.
% We will be developing experiments that will allow us to test the production rules
% that are output of our tool.

% On the theoretical side, the synthesis problem can be potentially
% modeled as minimization of a modified version of tree automaton~\cite{minTree},
% where the next move of the automaton additionally depends on the ancestors
% and siblings of a letter in a tree word.
% This modeling may allow us to apply many out of the box theoretical results
% for our synthesis problem.
% For example, we may view the rule synthesis problem as an automaton minimization problem, i.e.,
% finding the minimum set of transitions of the automaton such that the given set of
% words are accepted.
% The variation from the standard tree automaton has signification consequences such that
% the known efficient methods for the minimization are not applicable.
% Please note that we have applied an SMT solver on the problem.
% However, as far as we know, the synthesis problem is not known to be NP-complete. There may be more efficient methods.



%--------------------- DO NOT ERASE BELOW THIS LINE --------------------------

%%% Local Variables:
%%% mode: latex
%%% TeX-master: "main"
%%% End:
