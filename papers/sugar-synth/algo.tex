In this section, we present a method to solve the synthesis problem of finding
a set of production rules that produce a given set $\mu$ of molecules.
Our synthesis method \textsc{SugarSynth} is presented in algorithm~\ref{alg:sugar-synth}.
Here, we are considering only single compartment case.
The generalizations are discussed in appendix~\ref{sec:variations}.

% At a high level, our method is iterative.
% The method first synthesizes a set of rules $R$ that can produce the molecules in input set $\mu$, i.e., $\mu \subseteq P(S,R)$.
% However, there is also an additional requirement that a molecule
% that is not in $\mu$ is not producible by the synthesized rules $R$.
% Therefore, the method looks for the producible molecules that are not in $\mu$, called
% {\em counterexample molecule}.
% If there is no such molecule, our method terminates and reports
% the synthesized rules.
% If there is a molecule in $P(S,R) - \mu$, we add constraints that says
% that look for rules
% that do not generate
% the molecule and go for another iteration.
% Now let us look at the details of the method.


\begin{algorithm}[t]
  \caption{ \textsc{SugarSynth}($\mu$, $d$, $n$)}
  \textbf{Input     :} $\mu$ : molecules, $d$ : maximum rule depth, $n$ : number of rules \\
      \textbf{Output:}  $R$: synthesized rules
  \label{alg:sugar-synth}
  \begin{algorithmic}[1]
  \State $S$ := the set of monomers appear in $\mu$, $w$ := the maximum arity of a monomer in $S$
  \State $T$ := \textsc{MakeTemplatesRule}( $S$, $d$, $w$, $n$)
  \label{line:createRtemp}
  \State tCons := \textsc{RuleTemplateCorrectness}(T)
  \label{line:ruleCorr}
  \State Let $h$ is maximum of the heights of molecules in $\mu$
  \State $\hat{m}$ := \textsc{MakeTemplateMol}( $S$, $h$, $w$ )
  \label{line:createMtemp}
  \State $mCons := \textsc{MolTemplateCorrectness}(\hat{m},\mu)$
  \label{line:molCorr}
  \State $pCons$ := $\Land_{m \in \mu}$ \textsc{EncodeProduce}(m,T)
    \label{line:molenc}
  \State $nCons := \ltrue$
  %\State $negMol$ be set of molecules := $\emptyset$
  \While{$\ltrue$}
    \If{ $a$ = getModel( $tCons \land pCons \land nCons$ )}
    \label{line:posModel}
    \State $R$ := \textsc{readRules}($a$)
    \label{line:getR}
    \Else
       \State \textbf{ throw} Failed to synthesize the rules!
    \EndIf
    \State $rCons$ := \textsc{EncodeProduce}($\hat{m}$,Rs)
    \label{line:consNewR}
    \If{ $a$ = getModel( $mCons \land rCons$ )}
        \label{line:negModel}
        \State $m' :=$ \textsc{getNegMol}( $\hat{m}$, a)
        \State $nCons \landplus \forall \tau,cuts. \lnot \textsc{EncodeProduce}(m',T)$
        \label{line:encode-neg-mol}
        \label{line:negCons}
    \Else
       \State \Return synthesized rules $R$
    \EndIf

  \EndWhile
\end{algorithmic}
\end{algorithm}

%--------------------- DO NOT ERASE BELOW THIS LINE --------------------------

%%% Local Variables: 
%%% mode: latex
%%% TeX-master: "main"
%%% End: 


\subsection{\textsc{SugarSynth} in detail}

The method assumes that the input set $\mu$ is finite.
This is a reasonable assumption because
even if a set of rules can produce
an unbounded number of molecules,
no biology will exhibit an infinite set in a cell.
Our method
bounds the search space of production rules.
The method also takes two numbers as input:
$d$ is the maximum height of the learned rules
and 
$n$ is the maximum number of them.
% The parameters eliminate potential
% unbounded search space.
If the method fails to find production rules,
the user may call the method with larger parameters.
% However, we do not expect to find a large number of or sized rules
% because chemical reactions do not usually have many conditional
% on the precursors and do not apply large changes in a single step.
First, the method initializes $S$ with the set of monomers occurring
in $\mu$
and sets $w$ to be the maximum arity of any monomer in $S$.


We use templates to model the search space of rules.
A template is also a tree that has a depth and
the internal nodes of the tree have the same number of children.
Two variables label each node of the template.
One variable is for choosing the sugar at the node and the other is for describing
the `situation' of the node.
% The first variable at the node encodes choices of
% sugar monomers or their absence.
The domain of the first variables is $S \union \{\bot\}$.
Let $SVars$ be the unbounded set of variables with the domain.
We will use the pool of $SVars$ to add variables to the templates.

\begin{wrapfigure}{r}{0.37\textwidth}
    \vspace{-10mm}
        \center
  \begin{tikzpicture}[shorten >=1pt,thick,node distance=1cm,on grid,scale=0.5]
    \node[loc] (v0) {$v_0$};
    \node[sqloc, below right of=v0] (v1) {$v_e$};
    \draw [->] plot [smooth] coordinates { (v0.south)
          ($(v0.south)+(3mm,-2mm)$)
          ($(v0.south)+(10mm,1mm)$)
          % ($(v0.south)+(9mm,-6mm)$)
          % ($(v0.east)+(8mm,-8mm)$)
          (v1.north) };
    % \path[->] (v0) edge (v1);
    % \path[->,dashed] (v1) edge (v2);
    % \path[->] (v1) edge (v3);
    \fill[fill=gray,opacity=0.2] 
    (0,0) -- ++(-1.5cm,-1.7cm) -- ++(1.2cm,-1.2cm) -- ++(6cm,0cm) -- cycle;
    \fill[fill=gray,opacity=0.4] 
    (v1.center) -- ++(-1.5cm,-1.5cm) -- ++(3cm,0cm) -- cycle;
    \draw[dashed] (-1.5cm,-1.7cm) -- ++(-1.2cm,-1.2cm) -- ++(2.4cm,0cm) -- cycle;
  \end{tikzpicture}
  \caption{Parts of production rules in the rule templates.}
  \label{fig:rule-parts}
  \vspace{-8mm}
\end{wrapfigure}
A node in a production rule can be in four situations.
In Figure~\ref{fig:rule-parts}, we illustrate the situations.
The first situation is when a node position is in the expanding part.
The dark gray nodes are the expanding nodes.
The second situation is when a node position is not in the rule.
The dashed area are the absent nodes.
Among the matching nodes, we have two cases.
% , namely if the expanding
% nodes are the descendants of the matching node or not.
The third situation is when a node position is in the matching part
and has expanding descendants.
The nodes on the solid path from $v_0$ to the root $v_e$ of the expanding part are in the third situation.
% the nodes that are ancestors of expanding part among the matching nodes.
Finally, the fourth situation is the rest of node positions
in the matching part, which is the light gray area.
% The nodes in the rest of the light gray area are in the fourth situation.
A variable is mapped to a node to encode the four situations.
Let $K =\{Expand,Absent,MatchAns,Match\}$ be the set of symbols to
indicate the set of four situations.
Let $KVars$ be the unbounded set of variables with domain $K$. 
% We will use the pool of $KVars$ to add variables to the templates.
The templates are defined as follows.

\begin{df}
For given integers $d$ and $w$,
a {\em rule template} $t = (V, \nu, \kappa, C, v_{0r})$ is a labeled full tree with depth $d$ and 
each internal node has $w$ children, where 
$V$ is a set of nodes, $\nu : V \maps SVars$ maps nodes to distinct
sugar choice variables, 
$\kappa : V \maps KVars$ maps nodes to distinct situation variables,
$C : V \times \naturals \pmaps V$ maps the indexed children of nodes,
and
$v_{0r} \in V$ is the root of the tree.
\end{df}

For a node $v$ in a template if we assign $\kappa(v) = Absent$,
we call the node {\em absent}. Otherwise, we call the node {\em present}.
We will also be searching for the molecules that may be produced by the learned rules.
Therefore, we need to define the search space for the molecules.
We use templates for defining the search space.
We limit the template size using a parameter,
namely the height of the template.
\begin{df}
For given integers $h$ and $w$,
a {molecule template} $\hat{m} = (V, \nu, C, v_{0m})$ is a labeled full tree with height $h$ and 
each internal node has $w$ children, where 
$V$ is a set of nodes, $\nu : V \maps SVars$ maps nodes to sugar choice variables,
$C : V \times \naturals \pmaps V$ maps the indexed children of nodes,
and
$v_{0m} \in V$ is the root.
\end{df}


In the algorithm at line~\ref{line:createRtemp}, we call \textsc{MakeTemplatesRule}( $S$, $d$, $w$, $n$) to create $n$ templates for
height $d$ and children width $w$.
% We get $d$ and $n$ as input, and
% we let $w$ be the maximum arity of any sugar that occurs in $\mu$.
Since $w$ is the maximum arity of any sugar, we can map any node to any sugar.
% We also assume that templates
% returned by $\textsc{MakeTemplatesRule}$ do not share variables.
% We obtain a production rule from a template by assigning values
% to the variables of the template.
% Not all the assignments may result in a valid rule.
Next at line~\ref{line:ruleCorr}, we will construct constraints
that encode the set of valid rules. 
A valid assignment to the variables in
a template  $t = (V, \nu, \kappa, C, r)$ must
satisfy the following six conditions.
% The first two encode the basic agreement among variables.
% The third encodes the tree structure of the rule.
% The last three encode the relative positions of the four
% parts of the rule.
\begin{enumerate}
\item If a node is present, then it is labeled with a sugar.\\
  %Otherwise, we ignore the label.
$
\Land_{s \in S} \Land_{v \in V} (\kappa(v) \neq Absent \limplies \nu(v) \neq \bot )
$
\item
  % The number of children of a node should match with the arity of the labeled sugar.
  % The following formula states that

  The children that are at greater arity than that of the label
  are absent.\\
$
\Land_{s \in S} \Land_{i \in (arity(s),w] }\Land_{v \in V} (\nu(v) = s \limplies \kappa( C(v,i) ) = Absent )
$
\item If a node is present, then the parent of the node is also present.\\
  $
  \Land_{{\text{internal node } v \in V}}
  \Land_{i\in[1,w]} (\kappa(C(v,i) ) \neq Absent \limplies \kappa( v ) \neq Absent )  
  $
\item If a node is $Expand$, then its children are also $Expand$ if present.\\
  % The following constraints state that once one reached the expanding part, there are no descendants that are matching.
  $
\Land_{i\in[1,w]} \Land_{{\text{internal node } v \in V}} (\kappa(v) = Expand \limplies \kappa(C(v,i)) \in \{Expand,Absent\}   )
$
\item If a node is $Match$, then its children are also $Match$ if present.\\
  % Similarly, the following constraints state that once one reached the matching part without expanding descendants, there are no descendants that are expanding.
$
\Land_{i\in[1,w]} \Land_{{\text{internal node } v \in V}} (\kappa(v) = Match \limplies \kappa(C(v,i)) \in \{Match,Absent\}   )
$
\item If a node is $MatchAns$, then exactly one child is $MatchAns$ or $Expand$.\\
  % The following constraints encode that there is exactly one subtree that is expanding part
  % by stating that there is no branching in $MatchAns$ nodes and they form a path.
\mbox{}\hspace{-5mm}  
$
\Land_{v \in V} (\kappa(v) = MatchAns \limplies \sum_{i\in[1,w]}( \kappa(C(v,i)) \in \{MatchAns,Expand\}) = 1
$
\end{enumerate}

The call to $\textsc{RuleTemplateCorrectness}$ at
line~\ref{line:ruleCorr} creates the above constraints
and stores them in $tCons$.
% Next, 
At line~\ref{line:createMtemp}, we use the call to \textsc{MakeTemplateMol}($S$, $h$, $w$) to
create a molecule template of height $h$ and children width $w$.
%
Our method searches for unwanted producible molecules up to the height of $h$,
which we set to the maximum of the heights of molecules in $\mu$.
%
The choice of $h$ is arbitrary.
% %
% We assume if the rules could have produced a molecule with height less than $h$,
% then it must be in $\mu$.
% On the other hand, 
% We ignore the potential formation of the large size of molecules.
% %
% Since we observe only finite molecules in any wet experiment, we may assume
% there may be large producible molecules, but we did not observe them.
% However, this point is debatable and $h$ may be viewed as
% a parameter of the whole method.
%
Similar to the rule templates, not all assignments to molecule template variables
are valid.
We add the following conditions for
valid assignments for molecule template $\hat{m} = (V, \nu, C, \hat{v_0})$.
% The first  two conditions encode the tree structure of the molecule and
% matching the number of children with label arity for each node.
% The second encodes that the height of the molecule is large enough.
% The last encodes that new molecule must not be the same as any molecule in $\mu$.
\begin{enumerate}
\item If a node is present, then the parent of the nodes is also present.\\
  $
  \Land_{i\in[1,w]}\Land_{{\text{internal node } v \in V}} (\nu(C(v,i) ) \neq \bot \limplies \nu( v ) \neq \bot )  
  $
\item The children count of a node matches with the arity of the labeled sugar.\\
  $
  \Land_{s \in S} \Land_{i \in (arity(s),w] }\Land_{v \in V} (\nu(v) = s \limplies \kappa( C(v,i) ) = Absent )
  $
% \item At least one of the leaves is present. This is for
%   avoiding the trivial solution.\\
%   $
%   \Lor_{ v \in V \text{ and } depth(v) > d' } \nu( v ) \neq \bot,
%   $
%   where $d'$ is a number close to and smaller than $height(\hat{v_0})$.
%   This allows us to look for all molecules that are close
%   to height $h$.
%   We choose $d' = h-d$, where $d$ is the height of the rule templates.
  % This $d'$ ensures that we do not miss producible molecules that
  % can not have exact height $h$ but is slightly larger than $h$.
  % Then, there must be another producible
  % molecule that is a subtree of the molecule and has height in
  % between $h$ and $h-d$
  % because the maximum depth of a rule is $d$.
\item We find a molecule that is not in $\mu$.
  We encode $\Land_{{(\_,\_,\_, v_0)\in \mu}}  Neq(\hat{v_0},v_0)$, where
   predicate $Neq$ be recursively defined as follows.\\
  $Neq(v, v') :=\; \nu(v) \neq M(v') \lor
  \Lor_{{i \in [1,w]}} Neq( C(v,i), C(v',i ) )$\\
  $Neq(v,\bot) :=\;  \nu(v) \neq \bot$
  % \begin{align*}
  % Neq(v, v') :=\;&   \nu(v) \neq M(v') \lor
  % \Lor_{\mathclap{i \in [1,w]}} Neq( C(v,i), C(v',i ) )\\
  %   Neq(v,\bot) :=\;&  \nu(v) \neq \bot
  % \end{align*}
  % The constraint is
  % % \begin{align*}
  % % \end{align*}
  % In our implementation, 
  % we have an option to also ignore the subtrees of
  % the molecules of $\mu$.
\end{enumerate}
%
In our method, the call to $\textsc{MolTemplateCorrectness}$ at
line~\ref{line:molCorr} generates the above constraints and stores them in
$mCons$. 

We need to encode that the rules do generate the molecules
in $\mu$ and do not generate any other.
Using procedure $\textsc{EncodeProduce}$, we generate
the corresponding constraints.
We will discuss the procedure in-depth shortly.
Let us continue with~\textsc{SugarSynth}.
At line~\ref{line:molenc}, we call $\textsc{EncodeProduce}$ for
each molecule in $\mu$ and generate constraints $pCons$ stating that
the solutions of the templates will produce the molecules in $\mu$.

After producing the constraints $tCons$, $mCons$, and $pCons$,
the method enters in the loop.
It first checks the satisfiability of conjunction
$tCons \land pCons \land nCons$, where $nCons$ is $\ltrue$ in the first iteration
and will encode constraints related to counterexample molecules.
If the conjunction is unsatisfiable, there are no rules of
the input number and height,
and the method returns failure of synthesis.
If it is satisfiable, it constructs the rules at line~\ref{line:getR} from
%the satisfying assignment $a$
and stores them in $R$.

At line~\ref{line:consNewR}, we construct constraints $rCons$ again using
$\textsc{EnocdeProduce}$ that says that template molecule $\hat{m}$
is generated by rules $R$.
We check the satisfiability of $rCons \land mCons$.
If it is not satisfiable, we have found the rules that generate exactly
the molecules in $\mu$ and the loop terminates.
Otherwise, we use the satisfying assignment to create a
counterexample molecule $m'$.
At line~\ref{line:negCons}, we add constraints to $nCons$ stating that
all possible applications of template rules in $T$ must not produce $m'$.
We use shorthand $F \landplus G$ for $ F := F \land G$.
As we will see that $\textsc{EncodeProduce}$ introduces fresh variable
maps $\tau$ and $cuts$ in the encoding.
Since we negate the returned formula by $\textsc{EncodeProduce}$ 
and then we check the satisfiability.
% the encoding will not be correct
% due to the fresh variables.
Therefore, we need to introduce universal quantifiers for the
fresh variables.
We introduce universal quantifiers over $\tau$ and $cuts$
variables before adding to $nCons$.
% Intuitively, the universal quantifiers say that all possible
% application patterns of the learned rules do not produce $m'$.
Afterwords, the loop goes to the next iteration.
Since the domain of all the variables is finite, eventually the loop must
terminate.

\algnewcommand{\IIf}[1]{\State\algorithmicif\ #1\ \algorithmicthen}
\algnewcommand{\EndIIf}{\unskip\ \algorithmicend\ \algorithmicif}
\setlength{\textfloatsep}{0pt}%
\begin{algorithm}[H]
  \caption{\textsc{EncodeProduce}( $m$ : molecule (template), $T$ : rule (template) )}
  \label{alg:produce}
  \begin{algorithmic}[1]
    \State \Return $
    \Land_{v\in m}
    \Land_{t \in T}
    % \Land_{{\substack{m=(V,\_,\_,\_) \\ v\in V \\ t \in T }}}
    \left[
      rmatch(v) = t \land cuts(v) \limplies \Lor_{ {\ell \in [1,d)}} \textsc{EncodeP}( v, t, \ell)
      \right]
      $
  \end{algorithmic}
  \hrule
  \vspace{1ex}
  \hrule\vspace{2pt}
  \textsc{EncodeP}( $v$, $t = (\_,\nu,\kappa,v_{0r})$, $\ell$)\hfill\mbox{}
  \vspace{2pt}\hrule
  \begin{algorithmic}[1]
    \IIf{$ancestor(v,\ell) = \bot$} \Return $\lfalse$ \EndIIf
    \State $mark := \tau(v)$,\; $c := \ltrue$, \; $v_r := v_{0r}$, \; $i = \ell$
    \While{$i > 0$}
    \label{line:encodep-while}
    \State $v' := ancestor(v,i)$
    \State $c \landplus  \kappa(v_r) = MatchAns \land \nu(v_r) = M(v') \land \tau(v') < mark$
    \label{line:encodep-ans-match}
    \For{ $j \in [1,NumberChildren(v')]$ }
    \label{line:encodep-sub-for-loop}
    \If{$C(v',j) = ancestor(v,i-1)$}
       \label{line:encodep-jth-child-cond}
       \State $v_r' := C(v_r, j)$ 
       \label{line:encodep-next-vr}
    \Else
       \State $c \landplus \textsc{MatchTree}(C(v',j), C(v_r, j),mark, \lfalse)$
       \label{line:mtree}
    \EndIf
    \EndFor
    \State $v_r := v_r'$,\; $i := i - 1$
    \label{line:encodep-update-vr}
    \EndWhile
    \label{line:encodep-end-whileloop}
    \State $c \landplus  \kappa(v_r) = Expand \land \textsc{MatchTree}(v, v_r,mark,\ltrue) \land \textsc{MatchCut}(v, v_r,\lfalse)$
    \label{line:encodep-vr-expand}
    % % \State $c \landplus   $
    % \label{line:encodep-mtree-expand}
    % % \State $c \landplus  $
    % \label{line:mcut}
    \State \Return $c$
  \end{algorithmic}
  \hrule
  \vspace{1ex}
  \hrule\vspace{2pt}
  \textsc{MatchTree}( $v$, $v_r$, $mark$, $isExpand$ )\hfill\mbox{}
  \vspace{2pt}\hrule
  \begin{algorithmic}[1]
    \IIf{$v_r = \bot$} \Return $\ltrue$ \EndIIf
    \label{line:matchtree-vr-absent}
    \IIf{$v = \bot$} \Return $\kappa(v_r) = Absent$ \EndIIf
    \label{line:matchtree-v-absent}
    \State $tCons := isExpand$ \;? $( mark \leq \tau(v) ): ( \tau(v) < mark )$
    % \If{$isExpand$}
    %   \State $tCons := ( mark \leq \tau(v) )$
    % \Else
    %   \State $tCons := ( \tau(v) < mark ) $
    % \EndIf
    \State $c := \kappa(v_r) \neq Absent \limplies tCons \land \nu(v_r)= M(v) $
    \label{line:matchtree-cons}
    % \For{ $i \in [1,NumberOfChidren(v)]$ }
    % \State\hspace{-2ex} $c \landplus \textsc{MatchTree}( C(v,i), C(v_r,i), mark, isExpand)$
    % \EndFor
    \State  $c \landplus \Land_{i \in [1,NumberOfChidren(v)]} \textsc{MatchTree}( C(v,i), C(v_r,i), mark, isExpand)$
    \label{line:matchtree-recurse}
    \State \Return $c$   
  \end{algorithmic}
  \hrule
  \vspace{1ex}
  \hrule\vspace{2pt}
  \textsc{MatchCut}( $v$, $v_r$, $ruleParentIsNotAbsent$ )\hfill\mbox{}
  \vspace{2pt}\hrule
  \begin{algorithmic}[1]
    \IIf{$v = \bot$} \Return $\ltrue$ \EndIIf
    \label{line:mcut-no-v}
    \IIf{$v_r = \bot$} \Return $parentIsNotAbsent \limplies cuts(v)$ \EndIIf
    \label{line:mcut-no-vr}
    \State $c := ruleParentIsNotAbsent \limplies ( \kappa(v_r) = Absent) = cuts(v)$
    \label{line:mcut-cons}
    % \For{ $i \in [1,NumberOfChidren(v)]$ }
    % \State \hspace{-1.5ex}$c \landplus \textsc{MatchCut}( C(v,i), C(v_r,i), \kappa(v_r) \neq Absent)$
    % \EndFor
    \State $c \landplus \Land_{i \in [1,NumberOfChidren(v)]}\textsc{MatchCut}( C(v,i), C(v_r,i), \kappa(v_r) \neq Absent)$
    \label{line:mcut-kids}
    \State \Return $cons$
  \end{algorithmic}      

\end{algorithm}

%--------------------- DO NOT ERASE BELOW THIS LINE --------------------------

%%% Local Variables:
%%% mode: latex
%%% TeX-master: "main"
%%% End:


\subsection{\textsc{EncodeProduce} in detail}


Now let us look at the encoding generated by $\textsc{EncodeProduce}$.
The process of production of molecules adds pieces of trees
one after another.
In order to show that a molecule is producible by a set of production rules,
we need to find the nodes where the production rules are applied,
the rules that are applied on the nodes, and
the order of the application of the rules.
% The encoding is like encoding bounded executions of programs.
To model the production due to the application of the rules,
we attach three maps to the molecule nodes.
\begin{itemize}
\item Let $cuts$ map each node to a Boolean variable
indicating the node is the point where a rule is applied to
expand the molecule.
We say points of the applications of the production
rules as {\em cuts} of the tree.
% All the nodes that are in between the cuts are added in a single step.

\item Let $rmatch$ map each node to a rule indicating
  the rule that is applied to expand at the node.
  We need to match a rule to a node if it is a cut point.
  % If a node is cut point, we need to the rules that added the nodes  
\item Let $\tau$ map each node to an integer variable
  indicating the time point when the node was added to the molecule.
  Since already added nodes in a molecule decide what can be added later,
  we need to record the order of the addition.
  % If we do not capture the order in our encoding, we may wrongly declare
  % that a molecule is producible while it is not due to the matching
  % constraints.
\end{itemize}

Algorithm~\ref{alg:produce} presents function $\textsc{EncodeProduce}$
that returns the encoding.
It takes a molecule $m$ and a set of rules $T$.
Both the inputs can be template or concrete.
% The function is tolerant of the variations, but
Our presentation assumes that the molecule is concrete and
the rules are templates.
This will cover the case when the $\textsc{EncodeProduce}$ is called at
line~\ref{line:molenc} in algorithm~\ref{alg:sugar-synth}.
However, at line \ref{line:consNewR} the molecule is a template
and the rules are concrete, which we will discuss later.
% We will discuss the handling of this situation at the end of this subsection.
$\textsc{EncodeProduce}$ uses help of three other supporting functions,
$\textsc{EncodeP}$, $\textsc{MatchTree}$, and $\textsc{MatchCut}$.
% All of them are presented in algorithm~\ref{alg:produce}.
%

$\textsc{EncodeProduce}$ returns constraints stating for each node $v$
and rule template $t$,
if $v$ is at a cut and $t$ is applied at $v$, then $t$ must match
at the node $v$.
We require $rmatch(v)$ to be equal to some rule.
% Therefore, if $v$ is a cut node, some template must be applied. 
Since we are matching with template rule, we do not know the
position of the root of expanding nodes.
We enumerate to all possible depths from $1$ to $d-1$ for finding
the root.
For each $\ell \in [1,d)$, we call $\textsc{EncodeP}( v, t, \ell)$
to construct the constraints encoding that $t$ is applied at $v$
and the depth of the root of the expanding nodes is at depth $\ell$.

In $\textsc{EncodeP}$, we can traverse up from $v$ for $\ell$ steps to
find the node to match the root of $t$.
Naturally, it returns $\lfalse$ if there
is no $\ell$th ancestor of $v$. % i.e., there is no match.
The variable $mark$ is the timestamp for node $v$.
% During matching, we will add constraints stating that the matching part of the
% template should match to the nodes added before $mark$
% and expanding part of the template should match with nodes that
% are added at or after $mark$.
The local variable $c$ collects the conjunction of constraints as they are
generated. % and $c$ is initially $\ltrue$.
The local variable $v_r$ is initially equal to the root $v_{0r}$ of $t$
and traverses the nodes in $t$.
The while loop at line~\ref{line:encodep-while} in $\textsc{EncodeP}$ starts with
the $\ell$th ancestor from $v$,
matches all the ancestors up to the parent of $v$.
As the loop traverses down, it also traverses $t$ along the matching
using variable $v_r$.
In each iteration, $v_r$ is updated to the $j$th child due to lines
\ref{line:encodep-next-vr} and \ref{line:encodep-update-vr}
if the ancestors of $v$ also traverse along with the $j$th child.
% (see if condition at line \ref{line:encodep-jth-child-cond}).

% Let us consider the body of the loop.
Let $v'$ be the $i$th ancestor at some iteration of the loop.
At line~\ref{line:encodep-ans-match}, the loop adds constraints stating that
the corresponding node $v_r$ in the template rule is
$MatchAns$, sugar matches in $v_r$ and $v'$, and
node $v'$ is added before $mark$.
The loop at line~\ref{line:encodep-sub-for-loop}, iterates over
children of $v'$.
If there is a child of $v'$ that is not along the path to
$v$, it is matched at line~\ref{line:mtree} with the corresponding
child of $v_r$ by calling $\textsc{MatchTree}$, which traverses
rule template and molecule and generate constraints to encode that their
nodes match.
We will discuss $\textsc{MatchTree}$ shortly.
% $\textsc{MatchTree}$ is also called with a third parameter
% $mark$ and a Boolean value as the fourth parameter.
% We will discuss the procedure shortly.
If the $j$th child of $v'$ is along the path to $v$,
we get node $v'_r$ for updating
$v_r$ for the next iteration of the while loop.
% After the while loop in $\textsc{EncodeP}$, $v_r$ is the
% candidate node for the root of expanding nodes.
After the while loop in $\textsc{EncodeP}$ at line \ref{line:encodep-vr-expand}, we declare $v_r$
is $Expand$,  match the node $v$
with the corresponding node $v_r$ in the template rule by calling
$\textsc{MatchTree}$, and also call to match the cut pattern at
the subtree of $v$ with the template rule.

$\textsc{MatchTree}$ is a recursive procedure and matches
sugar assignments between the molecule and the rule template.
If the rule template node $v_r$ does not exist, then
there is nothing to match and it returns $\ltrue$ at
line~\ref{line:matchtree-vr-absent}.
At line~\ref{line:matchtree-v-absent}, we encode if the molecule node $v$ does not exist, then the rule node must also
be flagged absent. %Otherwise, the match will fail.
Otherwise, we add constraints
that if $v_r$ is not absent in the template, then
the labels of $v$ and $v_r$ must match at line~\ref{line:matchtree-cons}.
At the same line, $\textsc{MatchTree}$ also adds constraints $tCons$ that
nodes are added in the molecule in the correct order.
The last two inputs of the function is timestamp $mark$ and a bit $isExpand$,
which tells us that the matching nodes should be added before or after $mark$.
The calls to $\textsc{MatchTree}$ from \textsc{EncodeP} use the
$isExpand$ appropriately.
% to indicate that nodes correspond to the
% matching part of the rule are added before and expanding part nodes
% are added later.
Afterwords at line~\ref{line:matchtree-recurse},
the procedure calls itself for the children of $v$ and $v_r$.
% to generate the same constraints for the subtree.

Once a rule is applied to a molecule, all the
nodes inside the expanding part are added together.
Therefore, there should be no cuts within the set of added nodes.
Furthermore, if there are children nodes below the added nodes,
they must be added due to the application of some other rule.
Therefore, the children are at the cut points.
We call the above requirement {\em cut pattern}. 
$\textsc{MatchCuts}$ is a recursive function over the trees, and matches
cut patterns between the molecule node $v$ and the rule template node $v_r$.
The cuts must occur whenever nodes of the rule template transition
from present to absent.
For helping to detect the transition,
the third parameter is a constraint that encodes that the parent of $v_r$
is absent or not.
% If $v$ does not exist, then there is no constraint to add therefore
% the procedure returns $\ltrue$ at line~\ref{line:mcut-no-v}.
% If $v_r$ does not exist, then it returns constraints encoding
% that if the parent of $v_r$ was not absent, then $v$ must be at a cut at line~\ref{line:mcut-no-vr}.
% If both $v$ and $v_r$ exist, then we add constraints
% that if the parent of $v_r$ is not absent, $v_r$ is
% flagged to be absent in the learned rule
% if and only if $v$ is the cut points at line~\ref{line:mcut-cons}.
% $\textsc{MatchCuts}$ recursively traverses the children of $v$ and $v_r$
% and generates the constraints for the cut pattern.
% The constraints added by the procedure are guarded by the presence of parents
% because we need to enable constraints only when there is a possibility of transition
% from presence to absence.
The call to $\textsc{MatchCuts}$ in \textsc{EncodeP} passes $\lfalse$ as the third input.
Even if the parent of $v_r$ is present,
$v_r$ is a cut point adding cut pattern constraints for the node will create inconsistency.
Therefore, we are passing $\lfalse$.

In the case when we call \textsc{EncodeProduce} with template molecule and
concrete rules.
We need to swap the roles of $\nu$ and $M$ at their occurrences
at line \ref{line:encodep-ans-match} in \textsc{EncodeP} and
line \ref{line:matchtree-cons} in \textsc{MatchTree}.
Furthermore, $\kappa(v_r)$ is a concrete value depending on the situation of $v_r$ in
the rule.
In the functions, the variable is to be replaced by the
evaluated value of $\kappa(v_r)$.

Our algorithm always terminates. At each iteration, at least one solution, i.e, set of
rules is discarded.
In fact, we discard many because in each iteration we reject production of a counterexample molecule, but this may discard whole range of rule sets. 
Since the set of all possible rules is finite, we
guarantee termination. In the worst case, the entire search space is explored. For a problem having 10 node types, 3 as maximum rule depth, 10 rules to be synthesized, 3 compartments
and fast-slow reactions, the search space can only be as large as $\approx 10^{70}$ rules.
Our templates are sufficiently expressive to cover all aspects of biology.
The viability of the synthesized rules can only be verified by conducting wet experiments.

% On the other hand,
% if $v_r$ is also present, then $v$ must not be any cut point.


%--------------------- DO NOT ERASE BELOW THIS LINE --------------------------

%%% Local Variables: 
%%% mode: latex
%%% TeX-master: "main"
%%% End: 
