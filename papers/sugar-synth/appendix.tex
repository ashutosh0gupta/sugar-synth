\section{Extended related work}
\label{sec:related}

% Success of formal methods
Formal methods have successful in applications to a vast range of
problems from the analysis of systems.
%
In particular,
the verification of hardware designs
using SAT solver based model-checking has been applied
to industrial-scale problems~\cite{biere1999symbolic2}.
%
The verification of software systems is a much harder problem, and
many methods like bounded model-checking~\cite{biere2003bounded},
abstract interpretation~\cite{lattice77}, and
counterexample guided abstraction refinement (CEGAR)~\cite{ClarkeCEGAR} have
been successful for the problem.
%

Synthesis of programs that implements a given behavior
has been a focus of research for a while~\cite{PnueliSynthesis}.
%
In recent years, the synthesis of programs using SAT/SMT solvers
has gained momentum.
%
The approach encodes the search of a program that exhibits a certain
behavior into a satisfiability problem.
%
The solvers attempt to find a solution to the satisfiability problem.
%
The solution is the synthesized program.
%
In~\cite{SrivastavaSynthesis,Solar-Lezama2005},
the method has been successfully applied
to find programs that satisfy quantified specifications like sorting
programs, and have missing ``holes'' and need proper implementations
for them.
%
In~\cite{exampleSynth},
a set of pairs of input and output examples is the specification of a program,
and the synthesis method searches for programs in a space defined by a
template~\cite{sygus}.


% Boolean networks

% Program synthesis
Since the formal methods have been exhibiting effectiveness in the vast
range of problems, the
systems biology community has been applying the methods for various
biological problems~\cite{fisher2007executable}.
The formal methods approach is distinctly different from the statistical
approach traditionally used in biology.
In formal methods, we aim to match precisely the expected behavior rather
than learn approximate artifacts from biological data.
The use of formal methods belongs to two broad categories:
the analysis of biological models and the synthesis of the models.

The key focus has been to model gene regulatory networks (GRNs), since
they are the core of the central dogma of biology.
Boolean networks are often used to model
GRNs to find stable points or attractors, i.e., nodes in the networks
where a system eventually reaches no matter where it starts~\cite{wang2012BooleanOverview}.
% In~\cite{boolean-networks-stabilty}, the networks are used to evaluate
% stability of GRNs under mutations.
The Boolean networks are not often sufficient to adequately model the behavior
of GRNs.
The continuous-time Markov chains (CTMCs) are used to bring in the aspects of
timing and probabilistic constraints to GRNs.
The transient behaviors of GRNs are also crucial for various aspects
of biology.
% For example, the decision making in some cells is based on
% response timing to stimulus~\cite{response-timing-paper}.
In~\cite{delayedCTMC}, a method based on
model-checking is used to estimate the time evolution of the probability
distributions over states of GRNs.


There are many biological processes where we can observe the system behavior,
but the exact mechanisms of the processes are not known.
%
Recently, the methods for formal synthesis are finding their applications in biology
\cite{dunn2014defining,xuPluripotency,booleanModelKarp13,paoletti2014analyzing}.
%
In~\cite{koksal2013synthesis},
GRNs with unknown interactions are modeled using
an automata-like model with missing components
and the specification is the variations of the behavior of the system under mutations.
Their approach uses a counterexample-guided inductive synthesis (CEGIS) based algorithm~\cite{cegis}.
CEGIS is a framework for synthesis. It first finds a program that may satisfy `samples' from
the specification. If the synthesized program satisfies the specifications, the CEGIS terminates.
Otherwise, the method learns a new sample where the program violates the specifications.
It adds the sample to the set and goes to the next iteration.
Typically, the constraint solvers find the programs during intermediate steps of CEGIS by
solving the generated queries.
%
Our method follows a similar pattern, where
a set of sampled observations is a specification, i.e., output molecules,
and we need to find the governing programs, i.e., production rules.
%
In a subsequent work~\cite{fisher2015synthesising}, a Boolean network model is used,
the functions attached to the nodes are considered unknown, and
a different kind of data sets provides the behavior specification.
In this approach, their earlier method is adopted to work in the new situation.
%

As far as we know, there has been no similar computation based analysis of glycan production rules.
%
However, there has been a theoretical analysis of the properties of the production
rules in~\cite{Jaiman2018}. The work identifies the conditions for the production
of a finite set of molecules and the cases of unique production methods.
%
In our work, we are taking the computational approach of synthesis from formal methods
instead of looking for the theoretical conditions.

\section{Variations of the synthesis problem}
\label{sec:variations}
We have presented a simplified version of
the biological problem.
% However, the biological problem has more details.
We have developed variations of the above method to support more realistic problems.

% \subsection{Compartments}

{\em Compartments : } The production rules may live in different ordered compartments.
% The compartments are ordered.
A molecule moves from one compartment to the next.
The rules of the current compartment apply to the molecule.
% We have presented the formal description of the compartments in
% section~\ref{sec:model}.
To support the compartments, we take an additional integer
input $k$ in $\textsc{SugarSynth}$ to indicate the maximum number
of compartments.
We construct each template rule with a new variable with domain $[1,k]$.
Let $v_r$ be a node of some template rule. We write $compart(v_r)$
for the compartment variable for the template.

We will alter $\textsc{EncodeProduce}$ and its sub-procedures to
ensure that they enforce the compartment order.
We need to encode that a rule is applied when all the
pattern nodes were added in the current or earlier compartment.
% For the encoding, we add a new map $compart$ that maps
% nodes to variables with domain $[1,k]$.
We modify the function $\textsc{MatchTree}$ by replacing
$tCons$ assignment by the following code.\\
\begin{minipage}{1.0\linewidth}
\begin{algorithmic}[1]
  \setcounterref{ALG@line}{line:matchtree-v-absent}
  \State $tCons := isExpand \;?\; ( mark \leq \tau(v)  \land compart(v_r) = compart(v) )$\par
  \mbox{}\qquad\qquad\hspace{10mm} $:( \tau(v) < mark  \land compart(v_r) \geq compart(v) )$
\end{algorithmic}
\end{minipage}
Similarly, we modify the function
$\textsc{EncodeP}$ by inserting the following line after~\ref{line:encodep-ans-match}.\\
\begin{minipage}{1.0\linewidth}
\begin{algorithmic}[1]
  \setcounterref{ALG@line}{line:encodep-ans-match}
  \State $c \landplus compart(v_r) \geq compart(v')$
\end{algorithmic}  
\end{minipage}
% Using the above constraints, we can divide the learned rules
% into compartments.

% \subsection{Compartment stay model}

% When molecules pass through compartments, they are expected to stay
% for a while in the compartments.
% %
% The length of the stay determines the number of rule applications
% occurring on the molecule.
% %
% In our method, we assume that the molecules may stay any length of time
% in a compartment, which may result in any number of applications of the rules.
% %
% % This allows the maximum number of molecules are produced for a given set of rules.
% %
% We can assume a whole range of {\em stay models} for the synthesis.
% %
% We have considered one more stay model, where a molecule stays
% in the compartment until no more rule applications are possible.
% %
% We call the stay model as {\em infinite stay}.

% To encode the infinite stay model, we need to add constraints that
% say when a molecule goes to the next compartment, no rule of
% the current compartment is applicable.
% %
% We add the following constraints in \textsc{EncodeProduce}.
% $$
% \Land_{v\in m}\Land_{i \text{ such that } C(v,i) \neq \bot}\Land_{t \in T}
% compart( v ) \leq compart(t) < compart(C(v,i)) 
% \limplies \Land noMatch(C(v,i),t)
% $$
% where constraint $noMatch(v,t)$ is collected in \textsc{EncodeP}
% by inserting the following code after the while loop and an update on $c$ before making the last two updates on $c$.
% $noMatch(v,t)$ is $\ltrue$
% before the call to \textsc{EncodeP}.
% \begin{algorithmic}[1]
%   \vspace{1ex}
%   \setcounterref{ALG@line}{line:encodep-vr-expand}
%   \State $nomatch(v,t) \landplus \lnot c$
%   \vspace{1ex}
% \end{algorithmic}
% The constraints state that for each node $v$ and template rule $t$
% if a rule adds node $v$ before or at the compartment of $t$
% and its $i$th child after the compartment of $t$,
% then $t$ must not be applicable at the $i$th child.
% The above constraints effectively encode that the molecule
% cannot expand at some node until all the relevant rules on
% earlier compartments are disabled.
% Note that if there is no child at the $i$th node, we are adding no constraints of disabling rules of future compartments.
% One may also add those as a requirement, depending on
% the interpretation of the stay model.

\paragraph{Fast and slow reactions:}
There is rate associated with chemical reactions. We abstract this by defining slow and fast rules.
The fast rules dominate the slow rule.
A slow rule can occur only when no other fast rule is able to extend the molecule in that compartment.
Let us define function $\textsc{EncodeP}^-$, which generates constraints as $\textsc{EncodeP}$ expect
line~\ref{line:encodep-vr-expand} is missing, i.e., we do not analyze expand part.
% During the stay of a molecule in a compartment, it could be observed that some reactions were dominating than others. This can be explained by characterizing reactions as either slow or fast.
% We can now define \textsc{Extend} recursively as follows:
% $  \textsc{Extends}(r, m) :=\;   Apply(m,r) \lor \Lor_{{i \in [1,w]}} \textsc{Extends}( r, C(m,i ) )$\\
% $\textsc{Extends}(r, \bot) :=\;  \nu(v) \neq \bot$\\
We now modify $\textsc{EncodeP}$ after~\ref{line:encodep-vr-expand} to support fast reactions:\\
\begin{minipage}{1.0\linewidth}
\begin{algorithmic}[1]
  % \vspace{1ex}
  \setcounterref{ALG@line}{line:encodep-vr-expand}
  \State $c \landplus \lnot \textsc{Fast}(v_r) \implies 
  \Land_{t \in T} (\textsc{Fast}(t) \implies \lnot\lor_{ {\ell \in [1,d)}} \textsc{EncodeP}^-( v, t, \ell))$
  % \vspace{1ex}
\end{algorithmic}
  % \vspace{-1mm}
\end{minipage}
Here, \textsc{Fast} sets the constraint on the rule to be fast.
A negative molecule which is a proper subtree of an input molecule will cease to be negative if there is any fast reaction that is able to extend it as fast reactions happen aggressively and can make partial molecules complete. Due to limited space, we will not present the exact constraints.

% Therefore, we modify the function $\textsc{SugarSynth}$ after~\ref{line:encode-neg-mol}:
% \begin{algorithmic}[1]
%   \setcounterref{ALG@line}{line:encode-neg-mol}
%   \State $nCons \landplus
%   \lnot  \Lor_{r \in R} \textsc{Fast}(r) \land \textsc{Extends}(r,m')$
% \end{algorithmic}
% Here, \textsc{Fast} constraints the rule to be fast. The constraints state that none of the rules should be both fast and able to extend the negative molecule received from the model.

{\em Unbounded molecules: }
We only observe a finite set of molecules in cells.
However, a set of production rules may be capable of producing an unboundedly large number of molecules.
In such cases, rules produce molecules the have repeating patterns of a subtree while rest of the tree being exactly same as one of the input molecules. The rules may be acceptable in some biological settings. We modify constraints to not declare such molecules as negative.
Let us suppose we have a repeat pattern of depth $d$ with $r$ repetitions.
Let $v$ be the node in template molecule where repetition has begin.
We define $\textsc{RepeatHeads}(v,r,d)$ that returns nodes $v_0,.... v_r$ such that
$v_1 = v$, $v_i$ is the $d$th ancestor of $v_{i+1}$.
% Those rules should in the same compartment and are slow.
Let us define \textsc{Repeat} and \textsc{Exact}, which encodes that the trees rooted at $v_{i}$s repeat.\\
$\textsc{Repeat}([v_0,...,v_r], v') :=  \textsc{Exact}(v',v_r,\bot) \land \Land_{i\in[0,r-1]} \textsc{Exact}( v_i, v_{i+1},v_{i+1})$\\
% $\textsc{Exact}(v,v') := (M(v) = M(v')) \land \Land_{i \in [1,w]} \textsc{Exact}(C(v,i),C(v',i))$ \\
% $\textsc{Exact}(\bot,v') := \bot, \textsc{Exact}(v,\bot) := \bot, \textsc{Exact}(\bot,\bot) := \bot$\\
$\textsc{Exact}(\bot,v',\_) := \lfalse, \textsc{Exact}(v,\bot,\_) := \lfalse,$\\
$ \textsc{Exact}(\bot,\bot,\_) := \ltrue,\textsc{Exact}(v_s,\_,v_s) := \ltrue$\\
$\textsc{Exact}(v,v',v_s) := (M(v) = M(v')) \land \Land_{i \in [1,w]} \textsc{Exact}(C(v,i),C(v',i),v_s)$\\
% $\textsc{Repeat}(v, v', r, d) :=  \textsc{Repeat}(v, ancestor(v'), r, d-1)$\\
% $\textsc{Repeat}(v, v', r, 0) := (compart(v) = compart(v')) \land \textsc{Repeat}(v, C^{2d}(v'), r-1, d)$ \\
% $\textsc{Repeat}(v, v', 0, d) :=  \textsc{ExactMatch}(v,C^{2d}(v'))$ \\
% $\textsc{Repeat}(\bot, \bot, r, d) :=  \top  \quad \textsc{Repeat}(v, \bot, r, d) :=  \bot
% \quad \textsc{Repeat}(\bot, v', r, d) :=  \bot $\\
% \begin{align*}
%   \textsc{Repeat}(m, m', r, d) :=\;&  \textsc{Repeat}(m, ancestor(m'), r, d-1)\\
%   \textsc{Repeat}(m, m', r, 0) :=\;& (compart(m) = compart(m')) \land \textsc{Repeat}(m, C^{2d}(m'), r-1, d) \\
%   \textsc{Repeat}(m, m', 0, d) :=\;&  \textsc{ExactMatch}(m,C^{2d}(m')) \\
%   \textsc{Repeat}(\bot, \bot, r, d) :=\;&  \top \\
%   \textsc{Repeat}(m, \bot, r, d) :=\;&  \bot \\
%   \textsc{Repeat}(\bot, m', r, d) :=\;&  \bot \\
%   \textsc{ExactMatch}(m,m') :=\;& (M(m) = M(m')) \land \Land_{i \in [1,w]} \textsc{ExactMatch}(C(m,i),C(m',i)) \\
%   \textsc{ExactMatch}(\bot,m') :=\;& \bot \\
%   \textsc{ExactMatch}(m,\bot) :=\;& \bot 
% \end{align*}
  % where $ancestor(m')$ is the parent node of $m'$ and $C^{2*d}$ is the application of $C(m',i)$, $2d$ times and $i$ comes from the reverse of the path traversed by m' to reach it's $d^{th}$ parent.
%   The constraints for allowing runaway reactions are added after~\ref{line:consNewR} to modify \textsc{SugarSynth}.\\
% \begin{algorithmic}[1]
%   \vspace{1ex}
%   \setcounterref{ALG@line}{line:consNewR}
%   \State $mCons \landplus \Lor_{1 \leq r \leq r_0} \Lor_{1 \leq d \leq d_0} \Lor_{m \in \mu} (M(m) \neq M(\hat{m})) \implies  \textsc{Repeat}(m, \hat{m}, r, d) $
%   \vspace{1ex}
% \end{algorithmic}
% where \textsc{Repeat} adds the constraints described above due to which runaway reactions are allowed.
We modify the third constraints of $ \textsc{MolTemplateCorrectness}(\hat{m},\mu)$, which encodes that negative molecules not in $\mu$. We change the definition of $Neq$ as follows.\\
$Neq(v, v') :=\; (\nu(v) \neq M(v') \land \lor_{r\in[1,r_0],d\in[1,d_0]}\textsc{Repeat}(\textsc{RepeatHeads}(v,r,d), v')) $\\
\mbox{}\hspace{30mm}$\lor \Lor_{{i \in [1,w]}} Neq( C(v,i), C(v',i ) ),$\\
where $d_0$ and $r_0$ are limits on the depth of the repeating subtrees and the number of repetitions respectively.
The change will accept molecules with repeating patterns as positive samples.

\paragraph{Non-monotonic rules :}
Some production rules can not be applied if another node is present in a sibling.
We call such rule non-monotonic because it may get disabled as the molecule grows.
This feature of rules help in producing an exact set of desired molecules.
We add an extra bit on each node of rule
template called $\textsc{HardEnds}$.
If the node is absent, its parent is present, and  $\textsc{HardEnds}$ bit is true, then
no node must be present in the matching pattern at the time of the application of the rule.
We modify the function $\textsc{MatchTree}$ by inserting the following constraints after~\ref{line:matchtree-cons}:
\begin{minipage}{1.0\linewidth}
\begin{algorithmic}[1]
  \vspace{1ex}
  \setcounterref{ALG@line}{line:matchtree-cons}
  \State $c \landplus \textsc{HardEnds}(v_r) \implies (mark \leq \tau(v))$
\end{algorithmic}  
\end{minipage}
The constraints state that if the applicable rule has \textsc{HardEnds}, then it has to be added at a time earlier than the current time of the molecule, effectively restricting the addition of further rules.


%--------------------- DO NOT ERASE BELOW THIS LINE --------------------------

%%% Local Variables:
%%% mode: latex
%%% TeX-master: "main"
%%% End:


\section{Experiments with variations of the synthesis problem}
\label{sec:ex-variants}
We conducted separate experiments for the variations of the synthesis problem - non-monotonic rules and unbounded molecules using synthetic data.
%
\begin{table}
  \centering
  \begin{tabular}{|c|c|c|c|c|c|c|}\hline
     & \#molecules& \#Rules & Rule depth & \#Compartments & success? & Time (in secs.) 
          \\\hline
    %%cf-mucin.sugar%%
         &   & 6  & 2 & 1 & Yes &  2.85 \\\cline{3-7}
    N1   & 6 & 6  & 2 & 5 & Yes & {\bf 1.07}  \\\cline{3-7}
         &   & 5  & 3 & 1 & No & 0.81  \\\hline
         
    %%horse-cg.sugar%%
         &   & 6  & 2 & 1 & Yes & 1.02  \\\cline{3-7}
    N2   & 6 & 6  & 2 & 5 & Yes & {\bf 0.76}  \\\cline{3-7}
         &   & 6  & 4 & 1 & Yes & 0.82 \\\hline
         
    %%sars-cov2.sugar%%     
         &   & 9  & 2  & 1  & Yes & 3.18  \\\cline{3-7}
    N3   & 9 & 8  & 2 & 1 & Yes & {\bf 2.39}  \\\cline{3-7}
         &   & 8  & 3 & 2 & Yes  &  16.69 \\\hline
         
    %%human-cg.sugar%%     
         &   & 6  & 3  & 1  & Yes & 73.61  \\\cline{3-7}
    R1   & 5 & 5  & 3 & 2 & Yes & {\bf 47.72}  \\\cline{3-7}
         &   & 4  & 4 & 1 & No  &  16.99 \\\hline
         
    % %%D4.sugar%     
         &   & 6  & 3 & 3 & Yes &  5.25 \\\cline{3-7}
    R2   & 8 & 4  & 3 & 2 & Yes & {\bf 3.44}  \\\cline{3-7}
         &   & 4  & 3 & 5 & Yes  &  4.70 \\\hline
         
    % %%D5.sugar%     
         &   & 7  & 3 & 2 & Yes &  6.85 \\\cline{3-7}
    R3   & 8 & 7  & 3 & 4 & Yes & {\bf 2.09}  \\\cline{3-7}
         &   & 6  & 3 & 1 & No  & 0.77  \\\hline
  \end{tabular}   
  \end{table}
 We applied our tool on the synthetic datasets N1, N2 and N3 which required some rules to have $\textsc{HardEnds}$. There is a clear trend of increasing time as the size of our dataset increases. Between N1 and N2 which have the same number of molecules in the dataset, N1 has bulkier molecules and hence the time taken is higher. Reducing the number of compartments increases the time slightly however is successful in both N1 and N2. By Reducing the number of rules to be learnt, the tool fails in N1 as it required a minimum of 6 rules. Rules produced in these datasets have $\textsc{HardEnds}$ in the appropriate places to ensure that the exact set of molecules can be made.
 %

  To demonstrate the working of our tool in case of runaway reactions or datasets with unbounded molecules, we applied our tool on the synthetic datasets R1, R2 and R3. The molecules in these datasets have repeating patterns and we observe the synthesized rules to capture this pattern. The constraints added as part of unbounded molecules take care not to declare the molecules created by the repeated application of these rules to be negative. The time taken for synthesizing the rules is particularly high for R1 as the molecules in the dataset are relatively bulky. Increasing the number of compartments or Increasing the number of rules to learn increases the time in R2. The search in the combinatorial space is expensive, hence the time taken is more. 
  
\section{Implementation optimizations}

% \subsection{Optimizations}
We have incorporated several optimizations to gain efficiency or reduce
the number of iterations in the method. Let us discuss some of the remarkable
optimizations.

\paragraph{Ordered templates}
Our method constructs a list of templates.
Since the templates are symmetric, there is a potential for unnecessary
iterations.
Let us suppose the method rejects a set of learned rules.
In the subsequent queries include the constraints that reject the set of rules,
the solvers may choose the same set of rules again by permuting the mapping
from the templates to the learned rule.
To avoid these iterations, we define an arbitrary total order over the rules.
We add ordering constraints stating that any assignments to the list of templates should produce
ordered rules with respect to the total order.

\paragraph{Quantifier instantiations}
The constraints for rejecting the counterexample molecule is universally quantified at line~\ref{line:negCons} in Algorithm~\ref{alg:sugar-synth}.
For any assignment, the solver needs to try sufficiently many instantiations of the quantifiers
to ensure that the query is not unsatisfiable.
The instantiations sufficiently slow down the solving process.
We assist the solver by also adding an instantiation of the quantifiers that is equal
to the values of the corresponding variables in the assignment $a$ at line~\ref{line:negModel}.
We observe that for some inputs only adding the instantiations and not the universally quantified
formula is more effective.

\paragraph{Constraints for counterexample molecules}
In the presentation of \textsc{SugarSynth}, we construct constraints of counterexample molecules
at line~\ref{line:consNewR} in each iteration.
The repeated work is impractical because the formula management system of Z3 will be overwhelmed by
the construction of many terms repeatedly. In our implementation, we construct the constraints
once outside the while loop.
We pass the templates as the second parameter instead of concrete rules to~\textsc{EncodeProduce}.
Later at the solving time in line~\ref{line:negModel}, constraints are added to assign values
of the template variables that were returned by the solver at line~\ref{line:posModel}.
Due to the assignments, the templates become concrete rules in the context of solving.

\section{Features to support in future}

\subsection{Full organism data vs single cell data}

The experiments that observe the set of glycan molecules are of two kinds. In one kind, we isolate a single cell type organism and identify all the glycans present in the cells. In our presentation, we have assumed this kind of source of data. However, the experiments are difficult to conduct.
In another and more convenient way, we smash the whole organism and identify all the glycans present in all the cells.
So the information that which glycans are coming from which
cells is unknown.
In this situation, the synthesis has another task to map the molecules
in $\mu$ to several cell types.
A glycan may be present in multiple cell types.

Our method is easily adapted to consider this kind of data.
We search for a small covering set of subsets of
$\powerset{\mu}$ such that each set in the cover allows a small
set of production rules.
The cover sets indicate the different types of cells in terms of
the presence of glycans.
This variation makes the problem particularly hard.
There can be a potentially large number of covering subsets.
So far, we have not encountered a large enough data set such that
we apply the variation effectively. 


\subsection{Incomplete data}

The experiments are imperfect. They may not detect all possible glycans
in a cell.
We may need to leave the possibility of allowing a few more molecules to be
producible beyond $\mu$.
We replace the no extra molecule constraint by a constraint that allows molecules
that are `similar' to molecules in $\mu$.
However, we could not imagine any measure of similarity that naturally stems from
biological intuition.
For now, our tool supports the count of the number of monomer differences
as a measure of similarity.


%--------------------- DO NOT ERASE BELOW THIS LINE --------------------------

%%% Local Variables:
%%% mode: latex
%%% TeX-master: "main"
%%% End:
