We have presented a simplified version of
the biological problem. However, in real settings, we often encounter many more variants which require us to support additional constraints to model the set of molecules and possible rules in that particular problem. Hence, we developed the following variations of the method described in the main text to support more realistic problems.
% However, the biological problem has more details.
% We have developed variations of the above method to support more realistic problems.

% \subsection{Compartments}

{\em Compartments : } The production rules may live in different compartments.
% The compartments are ordered.
A molecule moves from one compartment to the next.
% The rules of the current compartment apply to the molecule.
% We have presented the formal description of the compartments in
% section~\ref{sec:model}.
To support the compartments, we take an additional integer
input $k$ in $\textsc{SugarSynth}$ to indicate the maximum number
of compartments.
We construct each template rule with a new variable with domain $[1,k]$.
Let $v_r$ be a node of some template rule. We write $compart(v_r)$
for the compartment variable for the template.

We will alter $\textsc{EncodeProduce}$ and its sub-procedures to
ensure that they enforce the compartment order.
We need to encode that a rule is applied when all the
pattern nodes were added in the current or earlier compartment.
% For the encoding, we add a new map $compart$ that maps
% nodes to variables with domain $[1,k]$.
We modify the function $\textsc{MatchTree}$ by replacing
$tCons$ assignment by the following code.\\
\begin{minipage}{1.0\linewidth}
\begin{algorithmic}[1]
  \setcounterref{ALG@line}{line:matchtree-v-absent}
  \State $tCons := isExpand \;?\; ( mark \leq \tau(v)  \land compart(v_r) = compart(v) )$\par
  \mbox{}\qquad\qquad\hspace{10mm} $:( \tau(v) < mark  \land compart(v_r) \geq compart(v) )$
\end{algorithmic}
\end{minipage}
Similarly, we modify the function
$\textsc{EncodeP}$ by inserting the following line after~\ref{line:encodep-ans-match}.\\
\begin{minipage}{1.0\linewidth}
\begin{algorithmic}[1]
  \setcounterref{ALG@line}{line:encodep-ans-match}
  \State $c \landplus compart(v_r) \geq compart(v')$
\end{algorithmic}  
\end{minipage}
% Using the above constraints, we can divide the learned rules
% into compartments.

% \subsection{Compartment stay model}

% When molecules pass through compartments, they are expected to stay
% for a while in the compartments.
% %
% The length of the stay determines the number of rule applications
% occurring on the molecule.
% %
% In our method, we assume that the molecules may stay any length of time
% in a compartment, which may result in any number of applications of the rules.
% %
% % This allows the maximum number of molecules are produced for a given set of rules.
% %
% We can assume a whole range of {\em stay models} for the synthesis.
% %
% We have considered one more stay model, where a molecule stays
% in the compartment until no more rule applications are possible.
% %
% We call the stay model as {\em infinite stay}.

% To encode the infinite stay model, we need to add constraints that
% say when a molecule goes to the next compartment, no rule of
% the current compartment is applicable.
% %
% We add the following constraints in \textsc{EncodeProduce}.
% $$
% \Land_{v\in m}\Land_{i \text{ such that } C(v,i) \neq \bot}\Land_{t \in T}
% compart( v ) \leq compart(t) < compart(C(v,i)) 
% \limplies \Land noMatch(C(v,i),t)
% $$
% where constraint $noMatch(v,t)$ is collected in \textsc{EncodeP}
% by inserting the following code after the while loop and an update on $c$ before making the last two updates on $c$.
% $noMatch(v,t)$ is $\ltrue$
% before the call to \textsc{EncodeP}.
% \begin{algorithmic}[1]
%   \vspace{1ex}
%   \setcounterref{ALG@line}{line:encodep-vr-expand}
%   \State $nomatch(v,t) \landplus \lnot c$
%   \vspace{1ex}
% \end{algorithmic}
% The constraints state that for each node $v$ and template rule $t$
% if a rule adds node $v$ before or at the compartment of $t$
% and its $i$th child after the compartment of $t$,
% then $t$ must not be applicable at the $i$th child.
% The above constraints effectively encode that the molecule
% cannot expand at some node until all the relevant rules on
% earlier compartments are disabled.
% Note that if there is no child at the $i$th node, we are adding no constraints of disabling rules of future compartments.
% One may also add those as a requirement, depending on
% the interpretation of the stay model.

\paragraph{Fast and slow reactions:}
There is a rate associated with chemical reactions. We abstract this by defining slow and fast rules.
The fast rules dominate the slow rule.
A slow rule can occur only when no other fast rule is able to extend the molecule in that compartment.
Let us define function $\textsc{EncodeP}^-$, which generates constraints as $\textsc{EncodeP}$ expect
line~\ref{line:encodep-vr-expand} is missing, i.e., we do not analyze expand part.
% During the stay of a molecule in a compartment, it could be observed that some reactions were dominating than others. This can be explained by characterizing reactions as either slow or fast.
% We can now define \textsc{Extend} recursively as follows:
% $  \textsc{Extends}(r, m) :=\;   Apply(m,r) \lor \Lor_{{i \in [1,w]}} \textsc{Extends}( r, C(m,i ) )$\\
% $\textsc{Extends}(r, \bot) :=\;  \nu(v) \neq \bot$\\
We now modify $\textsc{EncodeP}$ after~\ref{line:encodep-vr-expand} to support fast reactions:\\
\begin{minipage}{1.0\linewidth}
\begin{algorithmic}[1]
  % \vspace{1ex}
  \setcounterref{ALG@line}{line:encodep-vr-expand}
  \State $c \landplus \lnot \textsc{Fast}(v_r) \implies 
  \Land_{t \in T} (\textsc{Fast}(t) \implies \lnot\lor_{ {\ell \in [1,d)}} \textsc{EncodeP}^-( v, t, \ell))$
  % \vspace{1ex}
\end{algorithmic}
  % \vspace{-1mm}
\end{minipage}
Here, \textsc{Fast} sets the constraint on the rule to be fast.
A negative molecule which is a proper subtree of an input molecule will cease to be negative if there is any fast reaction that is able to extend it as fast reactions happen aggressively and can make partial molecules complete. Due to limited space, we will not present the exact constraints.

% Therefore, we modify the function $\textsc{SugarSynth}$ after~\ref{line:encode-neg-mol}:
% \begin{algorithmic}[1]
%   \setcounterref{ALG@line}{line:encode-neg-mol}
%   \State $nCons \landplus
%   \lnot  \Lor_{r \in R} \textsc{Fast}(r) \land \textsc{Extends}(r,m')$
% \end{algorithmic}
% Here, \textsc{Fast} constraints the rule to be fast. The constraints state that none of the rules should be both fast and able to extend the negative molecule received from the model.

{\em Unbounded molecules: }
We only observe a finite set of molecules in cells.
However, a set of production rules may be capable of producing an unboundedly large number of molecules.
In such cases, rules produce molecules that have repeating patterns of a subtree while rest of the tree being exactly same as one of the input molecules. The rules may be acceptable in some biological settings. We modify constraints to not declare such molecules as negative.
Let us suppose we have a repeat pattern of depth $d$ with $r$ repetitions.
Let $v$ be the node in template molecule where repetition has begin.
We define $\textsc{RepeatHeads}(v,r,d)$ that returns nodes $v_0,.... v_r$ such that
$v_1 = v$, $v_i$ is the $d$th ancestor of $v_{i+1}$.
% Those rules should in the same compartment and are slow.
Let us define \textsc{Repeat} and \textsc{Exact}, which encodes that the trees rooted at $v_{i}$s repeat.\\
$\textsc{Repeat}([v_0,...,v_r], v') :=  \textsc{Exact}(v',v_r,\bot) \land \Land_{i\in[0,r-1]} \textsc{Exact}( v_i, v_{i+1},v_{i+1})$\\
% $\textsc{Exact}(v,v') := (M(v) = M(v')) \land \Land_{i \in [1,w]} \textsc{Exact}(C(v,i),C(v',i))$ \\
% $\textsc{Exact}(\bot,v') := \bot, \textsc{Exact}(v,\bot) := \bot, \textsc{Exact}(\bot,\bot) := \bot$\\
$\textsc{Exact}(\bot,v',\_) := \lfalse, \textsc{Exact}(v,\bot,\_) := \lfalse,$\\
$ \textsc{Exact}(\bot,\bot,\_) := \ltrue,\textsc{Exact}(v_s,\_,v_s) := \ltrue$\\
$\textsc{Exact}(v,v',v_s) := (M(v) = M(v')) \land \Land_{i \in [1,w]} \textsc{Exact}(C(v,i),C(v',i),v_s)$\\
% $\textsc{Repeat}(v, v', r, d) :=  \textsc{Repeat}(v, ancestor(v'), r, d-1)$\\
% $\textsc{Repeat}(v, v', r, 0) := (compart(v) = compart(v')) \land \textsc{Repeat}(v, C^{2d}(v'), r-1, d)$ \\
% $\textsc{Repeat}(v, v', 0, d) :=  \textsc{ExactMatch}(v,C^{2d}(v'))$ \\
% $\textsc{Repeat}(\bot, \bot, r, d) :=  \top  \quad \textsc{Repeat}(v, \bot, r, d) :=  \bot
% \quad \textsc{Repeat}(\bot, v', r, d) :=  \bot $\\
% \begin{align*}
%   \textsc{Repeat}(m, m', r, d) :=\;&  \textsc{Repeat}(m, ancestor(m'), r, d-1)\\
%   \textsc{Repeat}(m, m', r, 0) :=\;& (compart(m) = compart(m')) \land \textsc{Repeat}(m, C^{2d}(m'), r-1, d) \\
%   \textsc{Repeat}(m, m', 0, d) :=\;&  \textsc{ExactMatch}(m,C^{2d}(m')) \\
%   \textsc{Repeat}(\bot, \bot, r, d) :=\;&  \top \\
%   \textsc{Repeat}(m, \bot, r, d) :=\;&  \bot \\
%   \textsc{Repeat}(\bot, m', r, d) :=\;&  \bot \\
%   \textsc{ExactMatch}(m,m') :=\;& (M(m) = M(m')) \land \Land_{i \in [1,w]} \textsc{ExactMatch}(C(m,i),C(m',i)) \\
%   \textsc{ExactMatch}(\bot,m') :=\;& \bot \\
%   \textsc{ExactMatch}(m,\bot) :=\;& \bot 
% \end{align*}
  % where $ancestor(m')$ is the parent node of $m'$ and $C^{2*d}$ is the application of $C(m',i)$, $2d$ times and $i$ comes from the reverse of the path traversed by m' to reach it's $d^{th}$ parent.
%   The constraints for allowing runaway reactions are added after~\ref{line:consNewR} to modify \textsc{SugarSynth}.\\
% \begin{algorithmic}[1]
%   \vspace{1ex}
%   \setcounterref{ALG@line}{line:consNewR}
%   \State $mCons \landplus \Lor_{1 \leq r \leq r_0} \Lor_{1 \leq d \leq d_0} \Lor_{m \in \mu} (M(m) \neq M(\hat{m})) \implies  \textsc{Repeat}(m, \hat{m}, r, d) $
%   \vspace{1ex}
% \end{algorithmic}
% where \textsc{Repeat} adds the constraints described above due to which runaway reactions are allowed.
We modify constraints of $ \textsc{MolTemplateCorrectness}(\hat{m},\mu)$, which encodes that negative molecules are not in $\mu$. We replace the definition of $Neq$ as follows.\\
$Neq(v, v') :=\; (\nu(v) \neq M(v') \land \lor_{r\in[1,r_0],d\in[1,d_0]}\textsc{Repeat}(\textsc{RepeatHeads}(v,r,d), v')) $\\
\mbox{}\hspace{30mm}$\lor \Lor_{{i \in [1,w]}} Neq( C(v,i), C(v',i ) ),$\\
where $d_0$ and $r_0$ are limits on the depth of the repeating subtrees and the number of repetitions respectively.
The change will accept molecules with repeating patterns as positive samples.

\paragraph{Non-monotonic rules :}
Some production rules can not be applied if another node is present in a sibling.
We call such rules non-monotonic because it may get disabled as the molecule grows.
This feature of rules helps in producing an exact set of desired molecules.
We add an extra bit on each node of rule
template called $\textsc{HardEnds}$.
If the node is absent, its parent is present, and  $\textsc{HardEnds}$ bit is true, then
no node must be present in the matching pattern at the time of the application of the rule.
We modify the function $\textsc{MatchTree}$ by inserting the following constraints after~\ref{line:matchtree-cons}:
\begin{minipage}{1.0\linewidth}
\begin{algorithmic}[1]
  \vspace{1ex}
  \setcounterref{ALG@line}{line:matchtree-cons}
  \State $c \landplus \textsc{HardEnds}(v_r) \implies (mark \leq \tau(v))$
\end{algorithmic}  
\end{minipage}
The constraints state that if the applicable rule has \textsc{HardEnds}, then it has to be added at a time earlier than the current time of the molecule, effectively restricting the addition of further rules.


%--------------------- DO NOT ERASE BELOW THIS LINE --------------------------

%%% Local Variables:
%%% mode: latex
%%% TeX-master: "main"
%%% End:
